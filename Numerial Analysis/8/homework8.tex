\documentclass{article}

%change the margin of the paper
%\usepackage[legalpaper, margin=0.1in]{geometry}
%using the \substack
\usepackage{amsmath}
% \mathbbm{}
\usepackage{bbm}

% NOTE \pdv  partial differential equation
\usepackage{physics}

% hyperlink Here
\usepackage{hyperref}
\hypersetup{
    colorlinks=true,
    linkcolor=blue,
    filecolor=magenta,
    urlcolor=cyan,
}

% \cancelto{0}{A} to cross the A to 0
\usepackage{cancel}


% this is the package for block comment \begin{comment} and \end{comment}
\usepackage{verbatim}
\usepackage{imakeidx}
% For multiple rows in tabular environment.
\usepackage{multirow}
% use this package to strikeout the word /st{}
%the color package is for the \textcolor{red} to highlight the text
\usepackage{color,soul}
% to define newcolumntype and \arraybackslash
\usepackage{array}
% hyperref is to call /url. hyphen packege to avoid that the url is too long
\PassOptionsToPackage{hyphens}{url}\usepackage{hyperref}
%Todo list, \newlist \setlist...
\usepackage{enumitem,amssymb}
\newlist{todolist}{itemize}{2}
\setlist[todolist]{label=$\square$}

% for image inserting
\usepackage{graphicx}
\graphicspath{./Desktop/Homework/123/7/}
\usepackage{subfig}

% \iff \leqlsant
\usepackage{amssymb}



% Code block begin(lstlisting) and end(lstlisting)
\usepackage{listings}
\usepackage{color}

\definecolor{dkgreen}{rgb}{0,0.6,0}
\definecolor{gray}{rgb}{0.5,0.5,0.5}
\definecolor{mauve}{rgb}{0.58,0,0.82}


%NOTE: Change the "language" parameter here
\lstset{frame=tb,
  language=Matlab,
  aboveskip=3mm,
  belowskip=3mm,
  showstringspaces=false,
  columns=flexible,
  basicstyle={\small\ttfamily},
  numbers=none,
  numberstyle=\tiny\color{gray},
  keywordstyle=\color{blue},
  commentstyle=\color{dkgreen},
  stringstyle=\color{mauve},
  breaklines=true,
  breakatwhitespace=true,
  tabsize=3
}

% Macros
%%%%%%%%%%%% Text Color %%%%%%%%%%%%%%%
\definecolor{mypink1}{RGB}{219, 48, 233}
\definecolor{myred1}{RGB}{231, 76, 60}
\definecolor{myred2}{RGB}{203, 67, 53}
\definecolor{myblue1}{RGB}{52, 152, 219}
\definecolor{mygray}{gray}{0.6}

%% Table Style
\newcolumntype{C}{>{\center\arraybackslash}m{.70\columnwidth}}
\newcolumntype{Y}{>{\center\arraybackslash}m{2cm}}



%NOTE title Here
\title{Homework 8 Numerical Analysis}
\author{Hanyuan Zhu}
\date{}



%%%%%%%%%%%%% NOTE Begin of Document %%%%%%%%%%%%%%%%%%%%%%%%%%%%%%%%%%%%%%%
%%%%%%%%%%%%%%%%%%%%%%%%%%%%%%%%%%%%%%%%%%%%%%%%%%%%%%%%%%%%%%%%%%%%%%%%%%%

\begin{document}
\maketitle

\subsection*{ Question 1}

We are given 0 to 4th degrees of Legendre polynomials from referrence.
\begin{equation}
  \begin{cases}
    P_0(x) = 1 & (P_0(x), P_0(x)) = 2 \\
    P_1(x) = x & (P_1(x), P_1(x)) = \frac{3}{2} \\
    P_2(x) = \frac{1}{2}(3 x ^2 - 1) & (P_2(x), P_2(x)) = 0.4\\
    P_3(x) = \frac{1}{2} (5 x^3 -3 x) & (P_3(x), P_3(x)) = 0.285714  \\
    P_4(x) = \frac{1}{8} (35 x^4 -30 x^2 + 3)  & (P_4(x), P_4(x)) = \frac{2}{9} \\
  \end{cases}
\end{equation}


\paragraph{3.7 (b)} $f(x) = log(1 + x^2) $
\begin{equation}
  \begin{split}
    &\int_{-1}^{1} f(x) P_0(x) dx = \int_{-1}^{1} log(1 + x^2)dx  = -4 + \pi + log(4)  \\
    &\int_{-1}^{1} f(x) P_1(x) dx = \int_{-1}^{1} log(1 + x^2) x dx = 0\\
    &\int_{-1}^{1} f(x) P_2(x) dx = \int_{-1}^{1} log(1 + x^2) \frac{1}{2}(3 x ^2 - 1) dx = \frac{10}{3} - \pi \\
    &\int_{-1}^{1} f(x) P_3(x) dx = \int_{-1}^{1} log(1 + x^2) \frac{1}{2} (5 x^3 -3 x) dx = 0\\
    &\int_{-1}^{1} f(x) P_4(x) dx = \int_{-1}^{1} log(1 + x^2) \frac{1}{8} (35 x^4 -30 x^2 + 3) dx = \frac{5 \pi}{2} - \frac{118}{15} \\
  \end{split}
\end{equation}
Non-Zero terms of least square approximation,
\begin{equation}
  \begin{split}
    l(x) &= \frac{(f,P_0)}{(P_0,P_0)} + \frac{(f,P_2)}{(P_2,P_2)} P_2 (x) + \frac{(f,P_4)}{(P_4,P_4)} P_4 (x)\\
    &= \frac{\pi + log(4) -4 }{2} + (\frac{25}{6} - \frac{5\pi}{4})(3 x ^2 - 1) + (\frac{45 \pi}{32} - \frac{177}{40}) (35 x^4 -30 x^2 + 3)\\
    &= \frac{log(4)}{2} - \frac{2333}{120}+  \frac{191 \pi}{32} + \frac{1589}{16} x^2  - \frac{3381}{32}  x^4
  \end{split}
\end{equation}

\paragraph{3.7 (c)} $f(x) = tan^{-1} x $
\begin{equation}
  \begin{split}
    &\int_{-1}^{1} f(x) P_0(x) dx = \int_{-1}^{1} tan^{-1}x dx  = 0 \\
    &\int_{-1}^{1} f(x) P_1(x) dx = \int_{-1}^{1} x tan^{-1}x  dx \approx 0.5708\\
    &\int_{-1}^{1} f(x) P_2(x) dx = \int_{-1}^{1} tan^{-1}x \frac{1}{2}(3 x ^2 - 1) dx = 0\\
    &\int_{-1}^{1} f(x) P_3(x) dx = \int_{-1}^{1} tan^{-1}x \frac{1}{2} (5 x^3 -3 x) dx = -0.0228612\\
    &\int_{-1}^{1} f(x) P_4(x) dx = \int_{-1}^{1} tan^{-1}x \frac{1}{8} (35 x^4 -30 x^2 + 3) dx =0 \\
  \end{split}
\end{equation}

Non-Zero terms of least square approximation,
\begin{equation}
  \begin{split}
    l(x) &= \frac{(f,P_1)}{(P_1,P_1)} x + \frac{(f,P_3)}{(P_3,P_3)} P_3 (x)\\
    &\approx 0.5708 \frac{2}{3} x - \frac{0.0228612}{0.285714} \frac{1}{2} (5 x^3 -3 x)\\
    &= 0.500555x - 0.2000036 x^3
  \end{split}
\end{equation}

\subsection*{ Question 2}
\paragraph{a.}
We will show the orthogonality, $(Q_i,Q_j)=0, i \neq j $ by induction.
Given $Q_0 = M_0$,
\begin{equation}
  \begin{split}
    (Q_0,Q_1) &= (Q_0, M_1 - \frac{(M_1, Q_0)}{(Q_0, Q_0)}Q_0)\\
    &= (Q_0, M_1) - \frac{(M_1, Q_0)}{(Q_0, Q_0)}(M_0,Q_0)\\
    &= (Q_0, M_1) - \frac{(M_1, Q_0)}{(Q_0, Q_0)}(Q_0,Q_0)\\
    &= (Q_0, M_1) - (Q_0, M_1) =0
  \end{split}
\end{equation}

We assume $(Q_n, Q_j) = 0$, for all $ j < n$,
\begin{equation}
  \begin{split}
    (Q_n, Q_j) &= (M_n - \sum^{n-1}_{i=0} \frac{(M_n, Q_i)}{(Q_i, Q_i)}Q_i, Q_j)\\
    &= (M_n, Q_j) - \sum^{n-1}_{\substack{i=0 \\ i \neq j}} \frac{(M_n, Q_i)}{(Q_i, Q_i)}(Q_i, Q_j) - \frac{(M_n, Q_j)}{(Q_j, Q_j)}(Q_j, Q_j)\\
    &= - \sum^{n-1}_{\substack{i=0 \\ i \neq j}} \frac{(M_n, Q_i)}{(Q_i, Q_i)}(Q_i, Q_j) = 0
  \end{split}
\end{equation}
That is $ j < n, (Q_n, Q_j) = 0 \Leftrightarrow \sum^{n-1}_{\substack{i=0 \\ i \neq j}} \frac{(M_n, Q_i)}{(Q_i, Q_i)}(Q_i, Q_j) = 0 $

Now we want to show $(Q_{n+1}, Q_j) = 0$, for all $ j < n+1$,
\begin{equation}
  \begin{split}
    (Q_{n+1}, Q_j) = &(M_{n+1} - \sum^{n}_{i=0} \frac{(M_{n+1}, Q_i)}{(Q_i, Q_i)}Q_i, Q_j)\\
    \text{ Case 1 : If }& n \neq j,  \\
    = &(M_{n+1}, Q_j) - \sum^{n-1}_{\substack{i=0 \\ i \neq j}} \frac{(M_{n+1}, Q_i)}{(Q_i, Q_i)}(Q_i, Q_j) \\
    &- \frac{(M_{n+1}, Q_n)}{(Q_n, Q_n)}(Q_n, Q_j) - \frac{(M_{n+1}, Q_j)}{(Q_j, Q_j)}(Q_j, Q_j)\\
    = &(M_{n+1}, Q_j) - 0 - \frac{(M_{n+1}, Q_n)}{(Q_n, Q_n)} 0 - (M_{n+1}, Q_j)\\
    = &(M_{n+1}, Q_j) - (M_{n+1}, Q_j) = 0\\
    \text{ Case 2 : If }& n = j,  \\
    = &(M_{n+1}, Q_j) - \sum^{n-1}_{\substack{i=0}} \frac{(M_{n+1}, Q_i)}{(Q_i, Q_i)}(Q_i, Q_n) \\
    &- \frac{(M_{n+1}, Q_n)}{(Q_n, Q_n)}(Q_n, Q_j) \\
    = &(M_{n+1}, Q_j) - \sum^{n-1}_{\substack{i=0}} \frac{(M_{n+1}, Q_i)}{(Q_i, Q_i)}0 - \frac{(M_{n+1}, Q_n)}{(Q_n, Q_n)}(Q_n, Q_n) \\
    = &(M_{n+1}, Q_j) - (M_{n+1}, Q_j) = 0\\
  \end{split}
\end{equation}

Therefore, such $\{Q_n\}^{\infty}_{n=0}$ satisfies $(Q_i,Q_j)=0, i \neq j $.

\paragraph{b.}
To get $Q_1, Q_2, Q_3, Q_4$ , we are given $Q_0 = M_0 = 1, M_1=x, M_2= x^2 , M_3=x^3, M_4 = x^4$.

If $f(x)g(x)$ is odd function,
\begin{equation}
  \begin{split}
    (f(x),g(x)) &= \int^{1}_{-1} f(x)g(x) dx =  \int^{1}_{0} f(x)g(x) dx +  \int^{0}_{-1} f(x)g(x) dx\\
    &=  \int^{1}_{0} f(x)g(x) dx + \int^{0}_{-1} -f(-x)g(-x) dx\\
    &=  \int^{1}_{0} f(x)g(x) dx + \int^{0}_{-1} f(-x)g(-x) d(-x)\\
    &=  \int^{1}_{0} f(x)g(x) dx + \int^{0}_{1} f(x)g(x) dx\\
    &=  \int^{1}_{0} f(x)g(x) dx - \int^{1}_{0} f(x)g(x) dx = 0\\
  \end{split}
\end{equation}


\begin{equation}
  Q_1 = x - \cancelto{0}{(x,1)}\frac{1}{(1,1)} = x
\end{equation}

\begin{equation}
  \begin{split}
    Q_2 &= x^2 - \cancelto{0}{(x^2,x)} \frac{x}{(x,x)} - (x^2,1)\frac{1}{(1,1)}\\
    &= x^2 -  \int^{1}_{-1} x^2 dx \frac{1}{\int^{1}_{-1} dx}\\
    &= x^2 -  \frac{2}{3} \frac{1}{2} = x^2 -  \frac{1}{3}
  \end{split}
\end{equation}

\begin{equation}
  \begin{split}
    Q_3 &= x^3 - \cancelto{0}{(x^3, x^2 -  \frac{1}{3})} \frac{Q_2}{\Vert Q_2 \Vert^2} - (x^3,x)\frac{x}{\Vert x \Vert^2} - \cancelto{0}{(x^3,1)}\frac{1}{2}\\
    &= x^3 - (x^3,x)\frac{x}{\Vert x \Vert^2}\\
    &= x^3 - \frac{2}{5} \frac{3}{2} x = x^3 - \frac{3}{5}x
  \end{split}
\end{equation}

\begin{equation}
  \begin{split}
    Q_4 &= x^4 - \cancelto{0}{(x^4, x^3 - \frac{3}{5}x)}\frac{Q_3}{\Vert Q_3 \Vert^2} - (x^4, x^2 - \frac{1}{3})\frac{Q_2}{\Vert Q_2 \Vert^2} - \cancelto{0}{(x^4,x)}\frac{x}{\Vert x \Vert^2} - (x^3,1)\frac{1}{2}\\
    &= x^4 - (x^4, x^2 - \frac{1}{3})\frac{Q_2}{\Vert Q_2 \Vert^2} - (x^4,1)\frac{1}{2} \\
    &= x^4 - \int^{1}_{-1} x^6 - \frac{ x^4}{3}dx \frac{x^2 - \frac{1}{3}}{\int^{1}_{-1} (x^2 - \frac{1}{3})^2 dx} - \frac{1}{5}\\
    &= x^4 - (\frac{2}{7} - \frac{2}{15}) \frac{x^2 - \frac{1}{3}}{\frac{2}{5}- \frac{2}{9}} - \frac{1}{5} = x^4 - \frac{45}{8}\frac{16}{7 \times 15} x^2 - \frac{45}{8}\frac{16}{7 \times 15}\frac{1}{3} - \frac{1}{5}\\
    &= x^4 - \frac{6}{7}x^2 + \frac{2}{7} - \frac{1}{5} = x^4 - \frac{6}{7}x^2 + \frac{3}{35}\\
  \end{split}
\end{equation}

\paragraph{c.}
$$ Q_1(1) = 1$$
$$ Q_2(1) = c (1 -  \frac{1}{3}) = 1  \rightarrow c = \frac{3}{2} $$
$$ Q_3(1) = c (1 - \frac{3}{5}) \rightarrow c = \frac{5}{2} $$
$$ Q_4(1) = c (1 - \frac{6}{7} + \frac{3}{35}) \rightarrow c = \frac{35}{8} $$

\begin{equation}
  \begin{split}
    &Q_1 = x\\
    &Q_2 = \frac{3}{2} (x^2 -  \frac{1}{3})\\
    &Q_3 = \frac{5}{2}(x^3 - \frac{3}{5}x)\\
    &Q_4 = \frac{35}{8}(x^4 - \frac{6}{7}x^2 + \frac{3}{35})
  \end{split}
\end{equation}


\subsection*{ Question 3}
\paragraph{a.}
Let $ f(t) = (1-2xt+t^2)^{-\frac{1}{2}}$, then $ f'(t) = -\frac{1}{2}(1-2xt+t^2)^{-\frac{3}{2}}(-2x+2t) $.
The first two terms of maclaurin series of $f(t)$ is
$f(t) = f(0) + f'(0)t = 1 + x t$.

Thus
\begin{equation}
  \begin{split}
    P_{0}(x) = f(0) = 1 \\
    P_{1}(x) = f'(0) = x
  \end{split}
\end{equation}

This satisfies the formula in (4.115).

\paragraph{b.}
From part a., we have LHS $  f'(t) = (1-2xt+t^2)^{-\frac{3}{2}}(x-t) = \frac{x-t}{1-2xt+t^2} f(t)$,
and RHS $ f'(t) = \sum_{n=1}^{\infty} n P_{n}(x) t^{n-1}$.

Combine two equations above,

\begin{equation}
  \begin{split}
    &\sum_{n=1}^{\infty} n P_{n}(x) t^{n-1} =  \frac{x-t}{1-2xt+t^2} \sum_{n=0}^{\infty} P_{n}(x) t^{n} \\
    &(1-2xt+t^2)\sum_{n=1}^{\infty} n P_{n}(x) t^{n-1} =  (x-t) \sum_{n=0}^{\infty} P_{n}(x) t^{n} \\
    &\begin{split}
      \sum_{n=1}^{\infty} n P_{n}(x) t^{n-1} &- 2xt \sum_{n=1}^{\infty} n P_{n}(x) t^{n-1} t^2 \sum_{n=1}^{\infty} n P_{n}(x) t^{n-1} \\
      &= x \sum_{n=0}^{\infty} P_{n}(x) t^{n} -t \sum_{n=0}^{\infty} P_{n}(x) t^{n}\\
      \sum_{n=1}^{\infty} n P_{n}(x) t^{n-1} &- 2x \sum_{n=1}^{\infty} n P_{n}(x) t^{n} +  \sum_{n=1}^{\infty} n P_{n}(x) t^{n+1}  \\
      &= x \sum_{n=0}^{\infty} P_{n}(x) t^{n} -  \sum_{n=0}^{\infty} P_{n}(x) t^{n+1}
    \end{split}
  \end{split}
\end{equation}

Relabel the sum above,
\begin{equation}
  \begin{split}
    \sum_{n=0}^{\infty} (n+1) P_{n+1}(x) t^{n} - 2x \sum_{n=1}^{\infty} n P_{n}(x) t^{n} &+  \sum_{n=2}^{\infty} (n-1) P_{n-1}(x) t^{n} \\
    &= x \sum_{n=0}^{\infty} P_{n}(x) t^{n} -  \sum_{n=1}^{\infty} P_{n-1}(x) t^{n}
  \end{split}
\end{equation}

Note the $ t^0 $ and $ t^1$ terms on both side can be cancelled out by given $ P_2 (x) $ formula of Legendre polynomial.
    \begin{equation}
      \begin{split}
        &P_1 (x) + 2 tP_2 (x) - 2 xt P_1 (x) = x P_0 (x) + xt P_1 (x) - tP_0(x)\\
        &2t P_2 (x) = (3xt - 1) P_1(x) + (x - t) P_0 (x)\\
        &\text{Substitute } P_1 (x) = x, P_0 (x) = 1\\
        &2 P_2 (x) = 3x^2 - 1\\
      \end{split}
    \end{equation}

Then the rest terms , $n \geq 2$, can be written as
    \begin{equation}
      \begin{split}
        &\begin{split}
          \sum_{n=2}^{\infty} (n+1) P_{n+1}(x) t^{n} - 2x \sum_{n=2}^{\infty} n P_{n}(x) t^{n}& +  \sum_{n=2}^{\infty} (n-1) P_{n-1}(x) t^{n} \\
          &= x \sum_{n=2}^{\infty} P_{n}(x) t^{n} -  \sum_{n=2}^{\infty} P_{n-1}(x) t^{n} \\
        \end{split}\\
        & (n+1) P_{n+1}(x)  =  2xn P_{n}(x) +  (1-n) P_{n-1}(x) + x P_{n}(x) -  P_{n-1}(x)\\
        & (n+1) P_{n+1}(x)  =  (2n+1)x P_{n}(x) -n P_{n-1}(x)\\
        & P_{n+1}(x)  =  \frac{2n+1}{n+1}xP_{n}(x) -\frac{n}{n+1} P_{n-1}(x)\\
      \end{split}
    \end{equation}

Let $ n = 1$ , we have
    \begin{equation}
      \begin{split}
        & P_{2}(x)  =  \frac{3}{2}xP_{1}(x) - \frac{1}{2} P_{0}(x)\\
        & P_{2}(x)  =  \frac{3}{2}x^2 - \frac{1}{2}\\
      \end{split}
    \end{equation}

Thus $P_{2}(x), P_{1}(x), \text{ and } P_{0}(x)$ also satisfies the recursion. Therfore For all $n \geq 1 $, we have
$$ P_{n+1}(x)  =  \frac{2n+1}{n+1}xP_{n}(x) -\frac{n}{n+1} P_{n-1}(x) $$ .

\paragraph{c.}


First of all, in part a., we have show $P_0(x)$ and $P_1(x)$ defined here is the same as defined in Legendre polynomials.

Secondly, by using the generating function, we derive the triple recursion relation in part b.
$$ P_{n+1}(x)  =  \frac{2n+1}{n+1}xP_{n}(x) -\frac{n}{n+1} P_{n-1}(x) $$ .

This is the same triple recursion relation of Legendre polynomials.

Thus any $ P_{n} $ derive from the same triple recursion relation by the same $P_0(x)$ and $P_1(x)$ are identical.

\paragraph{d.}
When $x = 1$, we have $f(t) = (1 - t)^{-1}$.
The maclaurin series of $f(t)$ is
$$ f(t) = \sum^{\infty}_{n=0} t^n $$

From the RHS of generating function, we have,
$$ \sum^{\infty}_{n=0} t^n  =  \sum^{\infty}_{n=0} P_{n}(1) t^n $$
$$ \rightarrow P_{n}(1) = 1$$

\paragraph{e.}

The triple recursion formula

$$ (n+1) P_{n+1}(x) = (2n+1) x P_n (x) - n P_{n-1}(x) $$

We want to show
$$ P_n (x) = 2^n \sum_{k=0}^n \binom{n}{k} \binom{\frac{n+k-1}{2}}{n} x^k$$

To show it by mathematical induction, we have $ P_0 = 1 $ and $ P_1 = x$, Then
$$ \text{By triple recursive method }P_2  = \frac{(2+1)x^2 - 1 }{1+1} = \frac{3}{2} x^2 - \frac{1}{2} $$
By $P_n(x)$ formula
\begin{equation}
  \begin{split}
    P_2 &= 2^2 \sum_{k=0}^2 \binom{2}{k} \binom{\frac{2+k-1}{2}}{2} x^k \\
     &=  2^2 \left( \binom{2}{2} \binom{\frac{3}{2}}{2} x^2 + \binom{2}{1} \binom{\frac{2}{2}}{2} x^1 + \binom{2}{0} \binom{\frac{1}{2}}{2} x^0 \right)\\
     &=  2^2 \left( 1 * \frac{3}{8} x^2 + 2 *0 x^1 + 1 *(-\frac{1}{8}) x^0 \right)\\
     &=  \frac{3}{2} x^2 -\frac{1}{2} \\
  \end{split}
\end{equation}

Now we assume $$ P_n (x) = 2^n \sum_{k=0}^n \binom{n}{k} \binom{\frac{n+k-1}{2}}{n} x^k$$
and
$$ P_{n-1} (x) = 2^{n-1} \sum_{k=0}^{n-1} \binom{n-1}{k} \binom{\frac{n+k-2}{2}}{n-1} x^k$$

By triple recursion, we have
\begin{equation}
  \begin{split}
    &   (n+1)P_{n+1}(x) = (2n+1) x P_n (x) - n P_{n-1}(x)\\
    =& (2n+1) x  2^n \sum_{k=0}^n \binom{n}{k} \binom{\frac{n+k-1}{2}}{n} x^k - n 2^{n-1} \sum_{k=0}^{n-1} \binom{n-1}{k} \binom{\frac{n+k-2}{2}}{n-1} x^k\\
    =& (2n+1) 2^n \sum_{k=0}^n \binom{n}{k} \binom{\frac{n+k-1}{2}}{n} x^{k+1} - 2^{n-1} \sum_{k=0}^{n-1} \left( n\binom{n-1}{k}\right) \binom{\frac{n+k-2}{2}}{n-1} x^k\\
    =& (2n+1) 2^n \sum_{k=1}^{n+1} \binom{n}{k-1} \binom{\frac{n+k-2}{2}}{n} x^{k} - 2^{n-1} \sum_{k=0}^{n-1} \left( \binom{n}{k} (n-k)\right) \binom{\frac{n+k-2}{2}}{n-1} x^k\\
    =& (2n+1) 2^n \sum_{k=1}^{n+1} \left( \frac{k}{n+1} \right) \binom{n+1}{k} \binom{\frac{n+k}{2}-1}{n} x^{k}\\
    &- 2^{n}  \sum_{k=0}^{n-1}  \left(  \frac{(n-k+1)(n-k)}{2 (n+1)}\right) \binom{n+1}{k}   \binom{\frac{n+k}{2} - 1}{n-1} x^k\\
    =& (2n+1) 2^n \sum_{k=1}^{n+1} \left( \frac{ 2 k}{n+k} \right) \binom{n+1}{k} \binom{\frac{n+k}{2}}{n+1} x^{k}\\
    &- 2^{n}  \sum_{k=0}^{n-1}  \left(  \frac{(n-k+1)(n-k)n }{ (n+k) (n+1)}\right)  \binom{n+1}{k}   \binom{\frac{n+k}{2}}{n} x^k\\
    =& (2n+1) 2^n \sum_{k=1}^{n+1} \left( \frac{ 2 k}{n+k} \right) \binom{n+1}{k} \binom{\frac{n+k}{2}}{n+1} x^{k}\\
    &- 2^{n}  \sum_{k=0}^{n-1}  \left(  \frac{(n-k+1)(n-k)n }{ (n+k) (\frac{k-n}{2})}\right)  \binom{n+1}{k}   \binom{\frac{n+k}{2}}{n+1} x^k\\
    =& (2n+1) 2^{n+1} \sum_{k=1}^{n+1} \left( \frac{k}{n+k} \right) \binom{n+1}{k} \binom{\frac{n+k}{2}}{n+1} x^{k}\\
    &+ 2^{n+1}  \sum_{k=0}^{n-1}  \left(  \frac{(n-k+1)n }{n+k}\right)  \binom{n+1}{k}   \binom{\frac{n+k}{2}}{n+1} x^k\\
  \end{split}
\end{equation}

For terms from $x^1$ to $x^{n-1}$ in RHS of (2)
\begin{equation}
  \begin{split}
    &= 2^{n+1} \sum_{k=1}^{n-1} \left(\frac{ (2n+1) k}{n+k} + \frac{(n-k+1)n}{n+k} \right) \binom{n+1}{k} \binom{\frac{n+k}{2}}{n+1} x^{k}\\
    &= 2^{n+1} \sum_{k=1}^{n-1} \left(\frac{ (2n+1) k + (n-k+1)n}{n+k} \right) \binom{n+1}{k} \binom{\frac{n+k}{2}}{n+1} x^{k}\\
    &= 2^{n+1} \sum_{k=1}^{n-1} \left(\frac{ (n + k)(n+1) }{n+k} \right) \binom{n+1}{k} \binom{\frac{n+k}{2}}{n+1} x^{k}\\
    &= 2^{n+1} \sum_{k=1}^{n-1} (n+1) \binom{n+1}{k} \binom{\frac{n+k}{2}}{n+1} x^{k}\\
    &=  (n+1) 2^{n+1} \sum_{k=1}^{n-1}\binom{n+1}{k} \binom{\frac{n+k}{2}}{n+1} x^{k}\\
  \end{split}
\end{equation}

For terms $x^{n}$ and $x^{n+1}$ in RHS of (2)
\begin{equation}
  \begin{split}
    &= (2n+1) 2^{n+1} \sum_{k=n}^{n+1} \left( \frac{k}{n+k} \right) \binom{n+1}{k} \binom{\frac{n+k}{2}}{n+1} x^{k}\\
    &= (2n+1) 2^{n+1} \left( \frac{n}{2n} \right) \binom{n+1}{n} \binom{\frac{2n}{2}}{n+1} x^{n} + (2n+1) 2^{n+1} \left( \frac{n+1}{2n+1} \right) \binom{n+1}{n+1} \binom{\frac{2n+1}{2}}{n+1} x^{n+1}\\
    &= (2n+1) 2^{n+1} \left( \frac{n}{2n} \right) \binom{n+1}{n} \binom{n}{n+1} x^{n} + (n+1)2^{n+1} \binom{n+1}{n+1} \binom{\frac{2n+1}{2}}{n+1} x^{n+1}\\
    &\text{Because }\binom{n}{n+1} = 0 \text{ in $x^n$ term, we can multiply any constant to the term}\\
    &= (n+1) 2^{n+1}  \binom{n+1}{n} \binom{n+n}{n+1} x^{n} + (n+1)2^{n+1} \binom{n+1}{n+1} \binom{\frac{2n+1}{2}}{n+1} x^{n+1}\\
    &= (n+1) 2^{n+1} \sum_{k=n}^{n+1} \binom{n+1}{k} \binom{\frac{n+k}{2}}{n+1} x^k
  \end{split}
\end{equation}

For the term $x^{0}$  in RHS of (2)
\begin{equation}
  \begin{split}
    &= 2^{n+1} \left(  \frac{(n+1)n }{n}\right)  \binom{n+1}{0}   \binom{\frac{n}{2}}{n+1} x^0\\
    &= (n+1) 2^{n+1}  \binom{n+1}{0}   \binom{\frac{n}{2}}{n+1} x^0\\
  \end{split}
\end{equation}

To sum up (3) (4) and (5), the LHS of (2)
$$  (n+1)P_{n+1}(x) =(n+1) 2^{n+1} \sum_{k=0}^{n+1} \binom{n+1}{k} \binom{\frac{n+k}{2}}{n+1} x^k $$
$$  P_{n+1}(x) = 2^{n+1} \sum_{k=0}^{n+1} \binom{n+1}{k} \binom{\frac{n+k}{2}}{n+1} x^k $$

Therefore $ P_n (x) = 2^n \sum_{k=0}^n \binom{n}{k} \binom{\frac{n+k-1}{2}}{n} x^k$

\paragraph{f.}
We have
$$ P_n (x) = 2^n \sum_{k=0}^n \binom{n}{k} \binom{\frac{n+k-1}{2}}{n} x^k$$

\begin{equation}
  \begin{split}
    \pdv{}{x}  P_n (x) &= 2^n \sum_{k=1}^n k \binom{n}{k} \binom{\frac{n+k-1}{2}}{n} x^{k-1}\\
    (1-x^2)\pdv{}{x}  P_n (x) &= 2^n \sum_{k=1}^n k \binom{n}{k} \binom{\frac{n+k-1}{2}}{n} x^{k-1}\\
    &-2^n \sum_{k=1}^n k \binom{n}{k} \binom{\frac{n+k-1}{2}}{n} x^{k+1}\\
    \pdv{}{x} [(1-x^2)\pdv{}{x}  P_n (x) ]& = 2^n \sum_{k=2}^n k (k-1)\binom{n}{k} \binom{\frac{n+k-1}{2}}{n} x^{k-2}\\
    &-2^n \sum_{k=1}^n k (k+1) \binom{n}{k} \binom{\frac{n+k-1}{2}}{n} x^{k}\\
    \pdv{}{x} [(1-x^2)\pdv{}{x}  P_n (x) ]& = 2^n \sum_{k=0}^{n-2} (k+2) (k+1)\binom{n}{k+2} \binom{\frac{n+k+1}{2}}{n} x^{k}\\
    &-2^n \sum_{k=1}^n k (k+1) \binom{n}{k} \binom{\frac{n+k-1}{2}}{n} x^{k}\\
  \end{split}
\end{equation}

Note the terms as k = 0, n-1 and n,
\begin{equation}
  \begin{split}
    \text{k = 0 term: } &2^n (0+2) (0+1)\binom{n}{0+2} \binom{\frac{n+0+1}{2}}{n} x^{0}\\
    &= 2^n n(n-1) \binom{\frac{n+1}{2}}{n} = 2^n n(n-1) \frac{n+1}{1-n}\binom{\frac{n-1}{2}}{n}\\
    &= - n(n+1) 2^n \binom{\frac{n-1}{2}}{n}
  \end{split}
\end{equation}

\begin{equation}
  \begin{split}
    \text{k = n-1 term: }   &-2^n (n-1)n \binom{n}{n-1} \binom{n-1}{n} x^{n-1} = 0\\
    \text{k = n term: }   &-2^n n (n+1) \binom{n}{n} \binom{\frac{2n-1}{2}}{n} x^{n}\\
    &= - n (n+1) 2^n  \binom{n}{n} \binom{\frac{2n-1}{2}}{n} x^{n}\\
  \end{split}
\end{equation}

Note the terms from k = 1 to n-2 of equation (11),
\begin{equation}
  \begin{split}
    \pdv{}{x} & [(1-x^2)\pdv{}{x}  P_n (x) ] \\
    &= 2^n \sum_{k=1}^{n-2} (k+2) (k+1)\binom{n}{k+2} \binom{\frac{n+k+1}{2}}{n} x^{k}\\
    &-2^n \sum_{k=1}^{n-2} k (k+1) \binom{n}{k} \binom{\frac{n+k-1}{2}}{n} x^{k}\\
    &= 2^n \sum_{k=1}^{n-2} (n-k)(n-k-1) \frac{n+k+1}{2} \frac{2}{k+1-n} \binom{n}{k} \binom{\frac{n+k-1}{2}}{n} x^{k}\\
    &-2^n \sum_{k=1}^{n-2} k (k+1) \binom{n}{k} \binom{\frac{n+k-1}{2}}{n} x^{k}\\
    &= - 2^n \sum_{k=1}^{n-2}(n-k)(n+k+1) \binom{n}{k} \binom{\frac{n+k-1}{2}}{n} x^{k}\\
    &-2^n \sum_{k=1}^{n-2} k (k+1) \binom{n}{k} \binom{\frac{n+k-1}{2}}{n} x^{k}\\
    &= 2^n \sum_{k=1}^{n-2} (-(n-k)(n+k+1) -  k (k+1))\binom{n}{k} \binom{\frac{n+k-1}{2}}{n} x^{k}\\
    &= 2^n \sum_{k=1}^{n-2} (k^2- n^2 + k-n -  k - k^2)\binom{n}{k} \binom{\frac{n+k-1}{2}}{n} x^{k}\\
    &= - n( n +1)2^n \sum_{k=1}^{n-2} \binom{n}{k} \binom{\frac{n+k-1}{2}}{n} x^{k}\\
    &= - n( n +1)P_n (x)\\
  \end{split}
\end{equation}

\subsection*{ Question 4}
\subsubsection*{ 4a.}
\begin{equation}
  \begin{split}
    &(1-x^2)^n = \sum_{k=0}^{n}\binom{n}{k} (-x^2)^k = \sum_{k=0}^{n}\binom{n}{k} (-1)^k x^{2k} \\
    &\dv[n]{}{x} (1-x^2)^n = \sum_{k=0}^{n}\binom{n}{k}(-1)^k \dv[n]{}{x} x^{2k} = n! \sum_{k=\frac{n}{2}}^{n}\binom{n}{k} (-1)^k \binom{2k}{n} x^{2k-n}\\
  \end{split}
\end{equation}

Thus, the highet order term of $P_n(x)$ is $\frac{(-1)^n }{2^n} \binom{2n}{n} x^{n} \neq 0$, so  $P_n(x)$ is a polynomials with degree n.

\subsubsection*{ 4b.}

Given $ P_n(x) = \frac{1}{2^n n!}\dv[n]{(1-x^2)^n}{x}$ and $P_m(x) =  \frac{1}{2^m m!} \dv[m]{(1-x^2)^m}{x}$.

First of all, we will show $\dv[n-k]{(1-x^2)^n}{x} \vert^1_{-1} = 0, k \leq n$.

$$ \dv[n-k]{(1-x^2)^n}{x} \Big\vert^1_{-1} = \sum^{n-k}_{s=1} \sum^{s}_{k=1} (x-1)^{n-p} (x+1)^{n-s+p} DL((x-1)^n (x+1)^n) \Big\vert^1_{-1} = 0$$
where,  $DL((x^n y^n)$ is linear combination of some derivates of $x^n y^n$

\begin{equation}
  \begin{split}
    \int^{1}_{-1} & P_n(x) P_m(x) dx = \frac{1}{2^n n!} \frac{1}{2^m m!} \int^{1}_{-1}\dv[n]{(1-x^2)^n}{x} \dv[m]{(1-x^2)^m}{x} dx\\
    =& \frac{1}{2^n n!} \frac{1}{2^m m!} ( \cancelto{0}{\dv[n-1]{(1-x^2)^n}{x} \dv[m]{(1-x^2)^m}{x}\Big\vert^1_{-1}}\\
    &- \int^{1}_{-1}\dv[n-1]{(1-x^2)^n}{x} \dv[m+1]{(1-x^2)^m}{x} dx)\\
    &\text{Repeat the integral by parts until the following steps }\\
    =& \frac{1}{2^n n!} \frac{1}{2^m m!} ( \cancelto{0}{\dv[n-m-1]{(1-x^2)^n}{x} \dv[2m]{(1-x^2)^m}{x}\Big\vert^1_{-1}}\\
    &- \int^{1}_{-1}\dv[n-m-1]{(1-x^2)^n}{x} \dv[2m+1]{(1-x^2)^m}{x} dx)\\
  \end{split}
\end{equation}

We know $(1-x^2)^m$ is 2m degree polynomials of x, therefore  $\dv[2m+1]{(1-x^2)^m}{x} = 0$. That is
\begin{equation}
  \begin{split}
    \int^{1}_{-1} & P_n(x) P_m(x) dx = - \int^{1}_{-1}\dv[n-m-1]{(1-x^2)^n}{x} \cancelto{0}{\dv[2m+1]{(1-x^2)^m}{x}} dx = 0
  \end{split}
\end{equation}



\end{document}

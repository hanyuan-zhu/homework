\documentclass{article}

%change the margin of the paper
%\usepackage[legalpaper, margin=0.1in]{geometry}
%using the \substack
\usepackage{amsmath}
% \mathbbm{}
\usepackage{bbm}

% NOTE \pdv  partial differential equation
\usepackage{physics}

% hyperlink Here
\usepackage{hyperref}
\hypersetup{
    colorlinks=true,
    linkcolor=blue,
    filecolor=magenta,
    urlcolor=cyan,
}

% \cancelto{0}{A} to cross the A to 0
\usepackage{cancel}


% this is the package for block comment \begin{comment} and \end{comment}
\usepackage{verbatim}
\usepackage{imakeidx}
% For multiple rows in tabular environment.
\usepackage{multirow}
% use this package to strikeout the word /st{}
%the color package is for the \textcolor{red} to highlight the text
\usepackage{color,soul}
% to define newcolumntype and \arraybackslash
\usepackage{array}
% hyperref is to call /url. hyphen packege to avoid that the url is too long
\PassOptionsToPackage{hyphens}{url}\usepackage{hyperref}
%Todo list, \newlist \setlist...
\usepackage{enumitem,amssymb}
\newlist{todolist}{itemize}{2}
\setlist[todolist]{label=$\square$}

% for image inserting
\usepackage{graphicx}
\graphicspath{./Desktop/Homework/123/7/}
\usepackage{subfig}

% \iff \leqlsant
\usepackage{amssymb}



% Code block begin(lstlisting) and end(lstlisting)
\usepackage{listings}
\usepackage{color}

\definecolor{dkgreen}{rgb}{0,0.6,0}
\definecolor{gray}{rgb}{0.5,0.5,0.5}
\definecolor{mauve}{rgb}{0.58,0,0.82}


%NOTE: Change the "language" parameter here
\lstset{frame=tb,
  language=Matlab,
  aboveskip=3mm,
  belowskip=3mm,
  showstringspaces=false,
  columns=flexible,
  basicstyle={\small\ttfamily},
  numbers=none,
  numberstyle=\tiny\color{gray},
  keywordstyle=\color{blue},
  commentstyle=\color{dkgreen},
  stringstyle=\color{mauve},
  breaklines=true,
  breakatwhitespace=true,
  tabsize=3
}

% Macros
%%%%%%%%%%%% Text Color %%%%%%%%%%%%%%%
\definecolor{mypink1}{RGB}{219, 48, 233}
\definecolor{myred1}{RGB}{231, 76, 60}
\definecolor{myred2}{RGB}{203, 67, 53}
\definecolor{myblue1}{RGB}{52, 152, 219}
\definecolor{mygray}{gray}{0.6}

%% Table Style
\newcolumntype{C}{>{\center\arraybackslash}m{.70\columnwidth}}
\newcolumntype{Y}{>{\center\arraybackslash}m{2cm}}



%NOTE title Here
\title{Homework 9 Numerical Analysis}
\author{Hanyuan Zhu}
\date{}

%%%%%%%%%%%%% NOTE Begin of Document %%%%%%%%%%%%%%%%%%%%%%%%%%%%%%%%%%%%%%%
%%%%%%%%%%%%%%%%%%%%%%%%%%%%%%%%%%%%%%%%%%%%%%%%%%%%%%%%%%%%%%%%%%%%%%%%%%%

\begin{document}

\maketitle


\section*{Question 1}

Now we are trying to use the way for Trapezoidal Rule error to derive the error function of Simpson’s Rule.
$x_{n-1} + h = x_n$

\begin{equation}
  \begin{split}
    E^{S}(f) &= \int_{x_0}^{x_n} f(x) dx - \sum^{\frac{n-2}{2}}_{i=0} \int_{x_{2i}}^{x_{2i+2}} P_2 (x) dx = \sum^{\frac{n-2}{2}}_{i=0} \int_{x_{2i}}^{x_{2i+2}} f(x)-P_2 (x) dx \\
    &=  \sum^{\frac{n-2}{2}}_{i=0} \int_{x_{2i}}^{x_{2i+2}} \Psi_2 (x) \frac{f^{(3)}(c_x)}{3!} dx \\
    &=  \sum^{\frac{n-2}{2}}_{i=0} \frac{f^{(3)}(c_x)}{3!} \int_{x_{2i}}^{x_{2i+2}} \Psi_2 (x)  dx \\
    &=  \sum^{\frac{n-2}{2}}_{i=0} \frac{f^{(3)}(c_x)}{3!} \int_{x_{2i}}^{x_{2i+2}} (x - x_{2i})(x - x_{2i+1})(x - x_{2i+2}) dx \\
    &=  \sum^{\frac{n-2}{2}}_{i=0} \frac{f^{(3)}(c_x)}{3!} \int_{x_{2i}}^{x_{2i+2}} (x - x_{2i})(x - x_{2i} - h)(x - x_{2i} - 2h) d(x-x_i) \\
    &=  \sum^{\frac{n-2}{2}}_{i=0} \frac{f^{(3)}(c_x)}{3!} \int_{0}^{2h} x (x - h)(x - 2h) dx \\
    &=  \sum^{\frac{n-2}{2}}_{i=0} \frac{f^{(3)}(c_x)}{3!} \int_{0}^{2h} x^3 - 3hx^2 +2h^2x \quad dx \\
    &=  \sum^{\frac{n-2}{2}}_{i=0} \frac{f^{(3)}(c_x)}{3!} (\frac{x^4}{4} - hx^3 +h^2x^2) \Big\vert^{2h}_0 \\
    &=  \sum^{\frac{n-2}{2}}_{i=0} \frac{f^{(3)}(c_x)}{3!} \cancelto{0}{( 4h^4 - 8h^4 + 4 h^4)} = 0\\
  \end{split}
\end{equation}

That is the error will vanish in this manner.


\section*{Question 2}
\subsection*{2a. Taylor expansion at $x_1$}
\begin{equation}
  \begin{split}
    f(x) = f(x_1) + \frac{f'(x_1)}{1!}(x-x_1) + \frac{f''(x_1)}{2!} (x-x_1)^2 + \frac{f'''(x_1)}{3!} (x-x_1)^3 + \frac{f^{(4)} (c)}{4!} (x-x_1)^4
  \end{split}
\end{equation}

\subsection*{2b. Integrate Taylor expansiion}
\begin{equation}
  \begin{split}
      \int^{x_2}_{x_0} f(x) dx = \int^{x_2}_{x_0} & (f(x_1) + \frac{f'(x_1)}{1!}(x-x_1) + \frac{f''(x_1)}{2!} (x-x_1)^2 \\
      &+ \frac{f'''(x_1)}{3!} (x-x_1)^3 + \frac{f^{(4)} (c)}{4!} (x-x_1)^4) \quad dx\\
      = \int^{x_2}_{x_0} & (f(x_1) + \frac{f'(x_1)}{1!}(x-x_1) + \frac{f''(x_1)}{2!} (x-x_1)^2 \\
      &+ \frac{f'''(x_1)}{3!} (x-x_1)^3 + \frac{f^{(4)} (c)}{4!} (x-x_1)^4) \quad d(x-x_1)\\
      = \int^{h}_{-h} & (f(x_1) + \frac{f'(x_1)}{1!} x + \frac{f''(x_1)}{2!} x^2 + \frac{f'''(x_1)}{3!} x^3 + \frac{f^{(4)} (c)}{4!} x^4)dx\\
      &\text{ Odd functions will be cancelled out because of identical positive and negative area }\\
      = \int^{h}_{-h} & (f(x_1) + \frac{f''(x_1)}{2!} x^2 +  \frac{f^{(4)} (c)}{4!} x^4)dx\\
      =&  (f(x_1) x  + \frac{f''(x_1)}{3!} x^3 +  \frac{f^{(4)} (c)}{5!} x^5) \Big\vert^{h}_{-h}\\
      =&  2 f(x_1) h    + \frac{f''(x_1)}{3} h^3 +  2 \frac{f^{(4)} (c)}{5!} h^5 \\
  \end{split}
\end{equation}



\subsection*{2c.Function of  $f''(x_1)$ }
By Taylor expansion at $x_1$, we have
\begin{equation}
  \begin{split}
    f(x) = f(x_1) + \frac{f'(x_1)}{1!}(x-x_1) + \frac{f''(x_1)}{2!} (x-x_1)^2 + \frac{f'''(x_1)}{3!} (x-x_1)^3 + \frac{f^{(4)} (c)}{4!} (x-x_1)^4
  \end{split}
\end{equation}

substitute $x_0$ and $x_2$ into equation above,
\begin{equation}
  \begin{split}
    f(x_0) &= f(x_1) + \frac{f'(x_1)}{1!}(x_0-x_1) + \frac{f''(x_1)}{2!} (x_0-x_1)^2 + \frac{f'''(x_1)}{3!} (x_0-x_1)^3 + \frac{f^{(4)} (c)}{4!} (x_0-x_1)^4\\
    &= f(x_1) - f'(x_1) h + \frac{f''(x_1)}{2} h^2  - \frac{f'''(x_1)}{3!} h^3 + \frac{f^{(4)} (c)}{4!} h^4\\
  \end{split}
\end{equation}

\begin{equation}
  \begin{split}
    f(x_2) &= f(x_1) + \frac{f'(x_1)}{1!}(x_2-x_1) + \frac{f''(x_1)}{2!} (x_2-x_1)^2 + \frac{f'''(x_1)}{3!} (x_2-x_1)^3 + \frac{f^{(4)} (c)}{4!} (x_2-x_1)^4\\
    &= f(x_1) + f'(x_1) h + \frac{f''(x_1)}{2} h^2  + \frac{f'''(x_1)}{3!} h^3 + \frac{f^{(4)} (c)}{4!} h^4\\
  \end{split}
\end{equation}

\begin{equation}
  \begin{split}
    &f(x_0) + f(x_2) = 2 f(x_1) + f''(x_1) h^2  + \frac{f^{(4)} (c)}{12} h^4\\
    \rightarrow &f''(x_1)  = \frac{1}{h^2}(f(x_0) - 2 f(x_1) + f(x_2)) - \frac{f^{(4)} (c)}{12} h^2  \\
  \end{split}
\end{equation}


\subsection*{2d. Error of Simpson Rule }

\begin{equation}
  \begin{split}
    &\text{Result from 2b. }\\
    &\int^{x_2}_{x_0} f(x) dx  = 2 f(x_1) h + \frac{f''(x_1)}{3} h^3 + 2 \frac{f^{(4)} (c)}{5!} h^5 \\
%    & \rightarrow  2 h f(x_1)  = \int^{x_2}_{x_0} f(x) dx - \frac{f''(x_1)}{3} h^3 - 2 \frac{f^{(4)} (c)}{5!} h^5\\
    &\text{Formula of Simpson Rule }\\
    &\int^{x_2}_{x_0} P_2(x) dx = \frac{h}{3}(f(x_0)+4f(x_1)+f(x_2)) %= \frac{h}{3}(f(x_0)-6 f(x_1)+f(x_2)) + 2 h f(x_1)\\
  \end{split}
\end{equation}

Substitute $ f''(x_1)  = \frac{1}{h^2}(f(x_0) - 2 f(x_1) + f(x_2)) - \frac{f^{(4)} (c)}{12} h^2  $ into results from 2b.
\begin{equation}
  \begin{split}
    &\int^{x_2}_{x_0} f(x) dx  = 2 f(x_1) h + \frac{\frac{1}{h^2}(f(x_0) - 2 f(x_1) + f(x_2)) - \frac{f^{(4)} (c)}{12} h^2  }{3} h^3 + 2 \frac{f^{(4)} (c)}{5!} h^5 \\
    &\int^{x_2}_{x_0} f(x) dx  = 2 f(x_1) h + \frac{1}{3}(f(x_0) - 2 f(x_1) + f(x_2)) h - \frac{f^{(4)} (c)}{36} h^5   +  \frac{f^{(4)} (c)}{60} h^5 \\
    &\int^{x_2}_{x_0} f(x) dx  = \frac{1}{3}(f(x_0) +4 f(x_1) + f(x_2)) h  - \frac{f^{(4)} (c)}{90} h^5 \\
    &\int^{x_2}_{x_0} f(x) dx  = \int^{x_2}_{x_0} P_2(x) dx  - \frac{f^{(4)} (c)}{90} h^5 \\
    \rightarrow &\int^{x_2}_{x_0} f(x)- P_2(x) dx = - \frac{f^{(4)} (c)}{90} h^5
  \end{split}
\end{equation}

\subsection*{2e. Error function in Riemann Form and error estimation with large n}
Let $x_i + h = x_{i+1}$ for $i = 0,1,2,..., n$ , and $ h = \frac{x_n-x_0}{n}$.
When n is sufficiently large, on   $[x_{2i}, x_{2i+2}]$, $ f^{(4)}(x) \approx f^{(4)} (c_i) $. Then
\begin{equation}
  \begin{split}
    f^{(3)} (2i+2)- f^{(3)} (2i) &=\int^{x_{2i+2}}_{x_{2i}}  f^{(4)} (x) dx \\
    &\approx \int^{x_{2i+2}}_{x_{2i}}  f^{(4)} (c_i) dx\\
    &=  f^{(4)} (c_i) \int^{x_{2i+2}}_{x_{2i}}  dx = 2 h f^{(4)} (c_i)    \\
  \end{split}
\end{equation}

\begin{equation}
  \begin{split}
    \int^{x_n}_{x_0} f(x)- P_2(x) dx &= \sum_{i=0}^{\frac{n-2}{2}} \int^{x_{2i+2}}_{x_{2i}} f(x)- P_2(x) dx =   \frac{h^5}{4!}  \sum_{i=0}^{\frac{n-2}{2}} f^{(4)} (c_i)\\
    &= - \frac{h^4}{180}  \sum_{i=0}^{\frac{n-2}{2}} 2hf^{(4)} (c_i)\\
    &\approx - \frac{h^4}{180}  \sum_{i=0}^{\frac{n-2}{2}} (f^{(3)} (2i+2)- f^{(3)} (2i))\\
    &= - \frac{h^4}{180}  (f^{(3)}(x_n)- f^{(3)}(x_0))\\
  \end{split}
\end{equation}

\subsection*{2f.Show  Theorem 5.2.5}
Using the result from 2e, by changing $a = x_0$ and $b = x_n$. For any function f on intervel $[a,b]$,
we will get the asymptotic estimation of the error of Simpson Rule,
\begin{equation}
  \begin{split}
    E^{S}(f)\approx  - \frac{h^4}{180}  (f^{(3)}(b)- f^{(3)}(a))\\
  \end{split}
\end{equation}



\end{document}

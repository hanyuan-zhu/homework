\documentclass{article}

%change the margin of the paper
\usepackage[legalpaper, margin=1 in]{geometry}

% hyperlink Here
\usepackage{hyperref}
\hypersetup{
    colorlinks=true,
    linkcolor=blue,
    filecolor=magenta,
    urlcolor=cyan,
}

% this is the package for block comment \begin{comment} and \end{comment}
\usepackage{verbatim}
\usepackage{imakeidx}
% For multiple rows in tabular environment.
\usepackage{multirow}
% use this package to strikeout the word /st{}
%the color package is for the \textcolor{red} to highlight the text
\usepackage{color,soul}
% to define newcolumntype and \arraybackslash
\usepackage{array}
% hyperref is to call /url. hyphen packege to avoid that the url is too long
\PassOptionsToPackage{hyphens}{url}\usepackage{hyperref}
%Todo list, \newlist \setlist...
\usepackage{enumitem,amssymb}
\newlist{todolist}{itemize}{2}
\setlist[todolist]{label=$\square$}

% Code block begin(lstlisting) and end(lstlisting)
\usepackage{listings}
\usepackage{color}

\definecolor{dkgreen}{rgb}{0,0.6,0}
\definecolor{gray}{rgb}{0.5,0.5,0.5}
\definecolor{mauve}{rgb}{0.58,0,0.82}

\lstset{frame=tb,
  language=Python,
  aboveskip=3mm,
  belowskip=3mm,
  showstringspaces=false,
  columns=flexible,
  basicstyle={\small\ttfamily},
  numbers=none,
  numberstyle=\tiny\color{gray},
  keywordstyle=\color{blue},
  commentstyle=\color{dkgreen},
  stringstyle=\color{mauve},
  breaklines=true,
  breakatwhitespace=true,
  tabsize=3
}

% Macros
%%%%%%%%%%%% Text Color %%%%%%%%%%%%%%%
\definecolor{mypink1}{RGB}{219, 48, 233}
\definecolor{myred1}{RGB}{231, 76, 60}
\definecolor{myred2}{RGB}{203, 67, 53}
\definecolor{myblue1}{RGB}{52, 152, 219}
\definecolor{mygray}{gray}{0.6}

%%%%%%%%%%%%%%%% Tags  %%%%%%%%%%%%%%%%
\newcommand{\pylb}{\textcolor{blue}{Python \quad}}
\newcommand{\llb}{\textcolor{green}{Latex \quad}}
\newcommand{\qlb}{\textcolor{red}{Question \quad}}
\newcommand{\mlb}{\textcolor{mypink1}{Math \quad}}
\newcommand{\main}{\textcolor{myred2}{MAIN \quad}}

%%Section name
\newcommand{\todo}{\textcolor{myred1}{TODO}}
\newcommand{\notes}{\textcolor{myblue1}{NOTES}}

%% Table Style
\newcolumntype{C}{>{\center\arraybackslash}m{.70\columnwidth}}
\newcolumntype{Y}{>{\center\arraybackslash}m{2cm}}

% Title setting
\title{Homework 4}
\author{Hanyuan Zhu}
\date{October 2, 2018}

%%%%%%%%%%%%%%%%%%%%%%%%%%%%%%%%%%%%%%%%%%%%%%%%%%%%%%%%%%%%%%%%%%%%%%%%%%%%%%%%
%%%%%%%%%%%%%%%%%%%%%%%%%%%%%%%%%%%%%%%%%%%%%%%%%%%%%%%%%%%%%%%%%%%%%%%%%%%%%%%%
%%%%%%%%%%%%%%%%%%%%%%%%%%%%%%%%%%%%%%%%%%%%%%%%%%%%%%%%%%%%%%%%%%%%%%%%%%%%%%%%
%%%%%%%%%%%%%%%%%%%%%%%%%%%%%%%%%%%%%%%%%%%%%%%%%%%%%%%%%%%%%%%%%%%%%%%%%%%%%%%%
%%%%%%%%%%%%%%%%%%%%%%%%%%%%%%%%%%%%%%%%%%%%%%%%%%%%%%%%%%%%%%%%%%%%%%%%%%%%%%%%
%%%%%%%%%%%%%%%%%%%%%%%%%%%%%%%%%%%%%%%%%%%%%%%%%%%%%%%%%%%%%%%%%%%%%%%%%%%%%%%%
%%%%%%%%%%%%%%%%%%%%%%%%%%%%%%BEGINNING OF DOCUMENT %%%%%%%%%%%%%%%%%%%%%%%%%%%%
%%%%%%%%%%%%%%%%%%%%%%%%%%%%%%%%%%%%%%%%%%%%%%%%%%%%%%%%%%%%%%%%%%%%%%%%%%%%%%%%
%%%%%%%%%%%%%%%%%%%%%%%%%%%%%%%%%%%%%%%%%%%%%%%%%%%%%%%%%%%%%%%%%%%%%%%%%%%%%%%%
%%%%%%%%%%%%%%%%%%%%%%%%%%%%%%%%%%%%%%%%%%%%%%%%%%%%%%%%%%%%%%%%%%%%%%%%%%%%%%%%
%%%%%%%%%%%%%%%%%%%%%%%%%%%%%%%%%%%%%%%%%%%%%%%%%%%%%%%%%%%%%%%%%%%%%%%%%%%%%%%%
%%%%%%%%%%%%%%%%%%%%%%%%%%%%%%%%%%%%%%%%%%%%%%%%%%%%%%%%%%%%%%%%%%%%%%%%%%%%%%%%
%%%%%%%%%%%%%%%%%%%%%%%%%%%%%%%%%%%%%%%%%%%%%%%%%%%%%%%%%%%%%%%%%%%%%%%%%%%%%%%%
%%%%%%%%%%%%%%%%%%%%%%%%%%%%%%%%%%%%%%%%%%%%%%%%%%%%%%%%%%%%%%%%%%%%%%%%%%%%%%%%
\begin{document}

\maketitle

\section{Bisection Method}

The smallest positive root is \textbf{$0.6662394324925154$}

\subsubsection*{The bisection algorithm is below by Python:}
\begin{lstlisting}
import numpy as np
import math
#iteration counter: to count how many steps to reach the threshold value
i=0

#Define the target funciton
def f(x):
    return math.tan(x)-math.cos(x)

def randompostive(x):
    return np.multiply(np.random.random(),x)

## NOTE: Inital generator, range:[0,1), half open interval, choosing a number uniformally randomly
a = randompostive(1);
b = randompostive(1);

## Here guarantees f(b)<= 0 <= f(a)
while f(a)<0:
    b = a;
    a = randompostive(1);

while f(b)>0:
    b = randompostive(1);

#to print the initial range
print abs(a-b)

### Setting the threshold epsilon
epsilon_ = math.pow(10,-15)

### Begin the bisection iterations. # NOTE:  Return r as root.
if f(a)*f(b) != 0:
    while abs(a-b) > epsilon_ :
        c = (a+b)/2
        if f(c) > 0:
            a = c;
        elif f(c) < 0:
            b = c;
        else:
            a = b;
        i = i+1;
    r = (a+b)/2;
elif  f(a) == 0:
    r = a;
else:
    r = b;

# print the root = 0.6662394324925154
print r

# print the number of iteratoins to get the root 50
print i

\end{lstlisting}


\section*{Newton's Method}

The smallest positive root is \textbf{$0.6662394324925153$}


The basin of attraction of the smallest positive root is the largest continuous interval of $f'(x)$ which contains the root.
\begin{equation}
  f'(x) = 1 + \tan^2 x + \sin x
\end{equation}

We know $\sin x$ is continuous on $ \mathbb{R} $, and $\tan^2 x$ is continuous on $ [ -4/\pi , 4/\pi] $, therefore the basin of attraction is \hl{$ [ -4/\pi , 4/\pi] $}

\subsubsection*{The Newton's algorithm is below by Python:}
\begin{lstlisting}
import numpy as np
import math
#iteration counter: to count how many steps to reach the threshold value
i=0

def randompostive(x):
    return np.multiply(np.random.random(),x)

# here iter(x) is x - f(x)/f'(x)
def iter(x):
    f_x = math.tan(x)-math.cos(x);
    df_x = 1+math.pow(math.tan(x),2)+math.sin(x);
    return x - f_x/df_x

## NOTE: Inital generator, range:[0,5). Since we get smallest root from bisection around 1.57.
x_1 = 0

x_2 = randompostive(2);

### Setting the threshold epsilon
epsilon_ = math.pow(10,-15)

while abs(x_1-x_2) > epsilon_:
    x_1 = x_2;
    x_2 = iter(x_1);
    i = i+1;

# x_2= 0.6662394324925153
print x_2

# i = 9
print i
\end{lstlisting}
\end{document}

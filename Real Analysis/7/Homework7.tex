\documentclass{article}

%change the margin of the paper
%\usepackage[legalpaper, margin=0.1in]{geometry}
%using the \substack
\usepackage{amsmath}
% \mathbbm{}
\usepackage{bbm}

% NOTE \pdv  partial differential equation
\usepackage{physics}

% hyperlink Here
\usepackage{hyperref}
\hypersetup{
    colorlinks=true,
    linkcolor=blue,
    filecolor=magenta,
    urlcolor=cyan,
}

% \cancelto{0}{A} to cross the A to 0
\usepackage{cancel}


% this is the package for block comment \begin{comment} and \end{comment}
\usepackage{verbatim}
\usepackage{imakeidx}
% For multiple rows in tabular environment.
\usepackage{multirow}
% use this package to strikeout the word /st{}
%the color package is for the \textcolor{red} to highlight the text
\usepackage{color,soul}
% to define newcolumntype and \arraybackslash
\usepackage{array}
% hyperref is to call /url. hyphen packege to avoid that the url is too long
\PassOptionsToPackage{hyphens}{url}\usepackage{hyperref}
%Todo list, \newlist \setlist...
\usepackage{enumitem,amssymb}
\newlist{todolist}{itemize}{2}
\setlist[todolist]{label=$\square$}

% for image inserting
\usepackage{graphicx}
\graphicspath{./Desktop/Homework/123/7/}
\usepackage{subfig}

% \iff \leqlsant
\usepackage{amssymb}



% Code block begin(lstlisting) and end(lstlisting)
\usepackage{listings}
\usepackage{color}

\definecolor{dkgreen}{rgb}{0,0.6,0}
\definecolor{gray}{rgb}{0.5,0.5,0.5}
\definecolor{mauve}{rgb}{0.58,0,0.82}


%NOTE: Change the "language" parameter here
\lstset{frame=tb,
  language=Matlab,
  aboveskip=3mm,
  belowskip=3mm,
  showstringspaces=false,
  columns=flexible,
  basicstyle={\small\ttfamily},
  numbers=none,
  numberstyle=\tiny\color{gray},
  keywordstyle=\color{blue},
  commentstyle=\color{dkgreen},
  stringstyle=\color{mauve},
  breaklines=true,
  breakatwhitespace=true,
  tabsize=3
}

% Macros
%%%%%%%%%%%% Text Color %%%%%%%%%%%%%%%
\definecolor{mypink1}{RGB}{219, 48, 233}
\definecolor{myred1}{RGB}{231, 76, 60}
\definecolor{myred2}{RGB}{203, 67, 53}
\definecolor{myblue1}{RGB}{52, 152, 219}
\definecolor{mygray}{gray}{0.6}

%% Table Style
\newcolumntype{C}{>{\center\arraybackslash}m{.70\columnwidth}}
\newcolumntype{Y}{>{\center\arraybackslash}m{2cm}}



%NOTE title Here
\title{Homework 8 Numerical Analysis}
\author{Hanyuan Zhu}
\date{}



%%%%%%%%%%%%% NOTE Begin of Document %%%%%%%%%%%%%%%%%%%%%%%%%%%%%%%%%%%%%%%
%%%%%%%%%%%%%%%%%%%%%%%%%%%%%%%%%%%%%%%%%%%%%%%%%%%%%%%%%%%%%%%%%%%%%%%%%%%

\begin{document}
\section*{Question 3.20}
We want to show the Cauchy sequence $\{ p_n\}$ converges to p, by given its subsequence $\{ p_{n_k}\}$ converges to p.

Since $\{ p_n \}$ is Cauchy sequence, for any $\delta >0$, there exist such N that $ n > N \& m > N $, then $ d(p_m,p_n) < \delta $.

Also because subsequence $\{ p_{n_k}\}$ converges to p,  for any given $\delta $ above, there exist such $N'$ that $ k > N' \leq N $, then
$ d(d_{n_k},p) < \delta $.

We let $ \epsilon = 2 \delta$. Thus, For any $\epsilon >0$, you can construct such $N'$ as above that $ d(d_n,p) <  d(p_n,p_{N'}) + d(p_{N'},p) < \delta + \delta = \epsilon $.
$\{p_n\}$ converges to  p.

\section*{Question 3.21}

First of all, we will show there is no distinct points in  $ \cap^{\infty}_1 E_n$ if  $ \cap^{\infty}_1 E_n \neq \emptyset $.

If  $  p\in \cap^{\infty}_1 E_n$ and  $ q \in \cap^{\infty}_1 E_n$. Then $p \in \lim_{n \rightarrow \infty} E_n $ and $q \in \lim_{n \rightarrow \infty} E_n $.
Then $ 0 \leq d(p,q) \leq \lim_{n \rightarrow \infty} diam E_n = 0 $, that is $p = q$.


Secondly, we will show $ \cap^{\infty}_1 E_n$ has at least one element.

$E_n \supset E_{n+1}$ and $\lim_{n \rightarrow \infty} diam E_n = 0 $ imply $ E_n $ is Cauchy sequence for all n.
Because X is a complete space, that means $\{ p_m \} = E_n $ converges to p. Since $E_n$ is closed,  the limit point $p \in E_n$ for all n.
Therefore there at least exists $ p \in \cap^{\infty}_1 E_n$.

To sum up, $\cap^{\infty}_1 E_n$ consist exactly one point, which is the limit point of sequence $ E_n$.

\section*{Question 3.22}

In order to utilize the result from Problem 3.21, We want to construct a closed bounded shrinking sequence of sets, $\overline{E_n} \in G_n$

Since $ \{G_n\}$ is dense and open in X, the complement set $F_n = G_{n}^{c}$ is closed and has no interier points (otherwise such a point won't be a limit point of $G_n$).
Thus for any open set $U \subset X$, $U \not\subset F_n $ and $U \backslash F_n$ is a open set. %For any n, $ \cap_{i=1}^{n} (U \backslash F_n) =  U \backslash (\cup_{i=1}^{n} F_n)  $ is open.


If $x \in U \backslash F_1$, there is a $N_{r_1}(x) \subset U \backslash F_1$. We Let $E_1 = N_{\frac{r_1}{2}}(x)\subset N_{r_1}(x) \subset U \backslash F_n $.
We choose $r_n$ by letting $N_{r_n}(x) \subset E_{n-1} \cap U \backslash F_{n-1}$, and $E_n = N_{\frac{r_n}{2}}(x) $.
Then $\overline{E_{n+1}} = \overline{N_{\frac{r_{n+1}}{2}}(x)} \subset N_{r_{n+1}}(x) \subset E_{n} \subset \overline{E_{n}} $

Because of $ N_{r_{n+1}}(x) \subset E_n \subset N_{\frac{r_{n}}{2}}(x) $, then $r_{n+1} < \frac{r_n}{2} < \frac{1}{2^{n-1}} r_1$. Therefore, as $n \rightarrow \infty$, $ r_n \rightarrow 0 $ and $ \text{diam } \overline{E_n} \rightarrow 0 $.

We know
\begin{equation}
  \begin{split}
    \overline{E_n} &\subset N_{r_{n}}(x) \subset E_{n-1} \cap (U \backslash F_n) \\
    &\subset  E_{n-2} \cap (U \backslash F_{n-1}) \cap (U \backslash F_n)\\
    & \subset \cap_{i=1}^{n} (U \backslash F_i)  = U \backslash (\cup_{i=1}^{n} F_i)\\
    & = U \backslash (\cup_{i=1}^{n} G_{i}^{c}) = U \backslash (\cap_{i=1}^{n} G_{i})^{c}\\
    & = U \cap (\cap_{i=1}^{n} G_{i})
  \end{split}
\end{equation}

Thus
\begin{equation}
  \begin{split}
    \cap_{i=1}^{\infty} \overline{E_n} &\subset \cap_{i=1}^{\infty} (U \cap (\cap_{j=1}^{i} G_{j}))\\
    & = U \cap (\cap_{i=1}^{\infty} G_{i})
  \end{split}
\end{equation}

According to result from Problem 3.21, we know $\cap_{i=1}^{\infty} \overline{E_n} \neq \emptyset $, that is $ U \cap (\cap_{i=1}^{\infty} G_{i}) \neq \emptyset $, then $\cap_{i=1}^{\infty} G_{i} \neq \emptyset $.

%If $x \in U \backslash F_n$, for any n,  there is a $N_{r_n}(x) \subset U \backslash (\cup_{i=1}^{n} F_n)$. We Let $E_n = N_{\frac{r_n}{2}}(x)\subset N_{r_n}(x) \subset U \backslash (\cup_{i=1}^{n} F_n) $.
%Then $\overline{E_n} = \overline{N_{\frac{r_n}{2}}(x)} \subset N_{r_n}(x) \subset U \backslash (\cup_{i=1}^{n} F_n) $

\section*{Question 3.23}
By the hints,
$$ d(p_n, q_n) \leq d(p_n, p_m) + d(p_m, q_m)+ d(q_n, q_m)$$

$$ d(p_n, q_n) -  d(p_m, q_m) \leq d(p_n, p_m) + d(q_n, q_m)$$

$$ \vert d(p_n, q_n) -  d(p_m, q_m) \vert \leq d(p_n, p_m) + d(q_n, q_m)$$

Since $\{ p_n \} $ $\{ q_n \} $ are cauchy sequence, then for any $\frac{\epsilon}{2}> 0$, there exists $N_1$ and $N_2$ such that if $n > N_1 \text{ and } n>N_2$ $d(p_n, p_m) < \frac{\epsilon}{2}$, also
if $n > N_2 \text{ and } n>N_2$ $d(q_n, q_m) < \frac{\epsilon}{2}$.

Then we let $N = max(N_1, N_2)$, we have For any $\epsilon > 0$, if  $n > N \text{ and } n>N$, then  $ \vert d(p_n, q_n) -  d(p_m, q_m) \vert < \frac{\epsilon}{2}+ \frac{\epsilon}{2} = \epsilon $. So it is a Cauchy Sequenc.
Because $d(p_n, q_n) \in R$, and R is a complete metric space, therefore the  sequence $\{d(p_n, q_n)\}$ converges on R.


\section*{Question 3.24}
\subsection*{3.24 a}
1. $d(p_n, p_n)= \vert p_n - p_n \vert = 0$;\\
2. $d(q_n, p_n)=  \vert q_n - p_n \vert =  \vert p_n - q_n \vert = d(q_n, p_n)$;\\
3. By triangle inequality, let $\{ r_n \} $ be sequence in X, $d(p_n, r_n) \leq d(p_n, q_n) + d(q_n, r_n)$,
then $\lim_{n\rightarrow \infty} d(p_n, q_n) = 0 $ and $\lim_{n\rightarrow \infty} d(q_n, r_n) = 0 $ implies $\lim_{n\rightarrow \infty} d(p_n, r_n) = 0 $.

\subsection*{3.24 b}
In question 3.24a, we have showed that for equivalent sequences, $ \lim_{n\rightarrow \infty} d(p_n, p'_n) = 0$

$ \lim_{n\rightarrow \infty} d(p'_n, q'_n) \leq \lim_{n\rightarrow \infty} d(p_n, p'_n) + \lim_{n\rightarrow \infty} d(p_n, q_n) + \lim_{n\rightarrow \infty} d(q_n, q'_n) =  \lim_{n\rightarrow \infty} d(p_n, q_n)$

Again, using the triangle inequality from the otherside, we will show $\lim_{n\rightarrow \infty} d(p'_n, q'_n) = \lim_{n\rightarrow \infty} d(p_n, q_n)$.

$\Delta (P,Q) = \lim_{n\rightarrow \infty} d(p_n, q_n) = \lim_{n\rightarrow \infty} d(p'_n, q'_n)$. Therefore it is unchanged by replacing equivalent sequences.

\section*{Question 1}
\subsection*{1 a}

$f(x) =  \frac{1}{2} (x + \frac{\alpha}{x})$, then let $\dv{f(x)}{x} = \frac{1}{2} (1 - \frac{\alpha}{x^2})= 0$.  we have postive solution at $ x = \sqrt{\alpha}$.
Therefore $x_n> \sqrt{\alpha} $.

\subsection*{1 b}
Given $x_n > \sqrt{\alpha}$, we have
\begin{equation}
  \begin{split}
    x_{n+1} - x_n &= \frac{1}{2} (x_n + \frac{\alpha}{x_n})- x_n\\
    &= \frac{x_n^2 + \alpha - 2 x_n^2}{2 x_n}\\
    &= \frac{\alpha - 2 x_n^2}{2 x_n} < 0
  \end{split}
\end{equation}
Therefore the sequence is monotonically decreasing.



\end{document}

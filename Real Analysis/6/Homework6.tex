\documentclass{article}

%change the margin of the paper
%\usepackage[legalpaper, margin=0.1in]{geometry}
%using the \substack
\usepackage{amsmath}
% \mathbbm{}
\usepackage{bbm}

% NOTE \pdv  partial differential equation
\usepackage{physics}

% hyperlink Here
\usepackage{hyperref}
\hypersetup{
    colorlinks=true,
    linkcolor=blue,
    filecolor=magenta,
    urlcolor=cyan,
}

% this is the package for block comment \begin{comment} and \end{comment}
\usepackage{verbatim}
\usepackage{imakeidx}
% For multiple rows in tabular environment.
\usepackage{multirow}
% use this package to strikeout the word /st{}
%the color package is for the \textcolor{red} to highlight the text
\usepackage{color,soul}
% to define newcolumntype and \arraybackslash
\usepackage{array}
% hyperref is to call /url. hyphen packege to avoid that the url is too long
\PassOptionsToPackage{hyphens}{url}\usepackage{hyperref}
%Todo list, \newlist \setlist...
\usepackage{enumitem,amssymb}
\newlist{todolist}{itemize}{2}
\setlist[todolist]{label=$\square$}

% for image inserting
\usepackage{graphicx}
\graphicspath{./Desktop/Homework/123/7/}
\usepackage{subfig}

% \iff \leqlsant
\usepackage{amssymb}



% Code block begin(lstlisting) and end(lstlisting)
\usepackage{listings}
\usepackage{color}

\definecolor{dkgreen}{rgb}{0,0.6,0}
\definecolor{gray}{rgb}{0.5,0.5,0.5}
\definecolor{mauve}{rgb}{0.58,0,0.82}


%NOTE: Change the "language" parameter here
\lstset{frame=tb,
  language=Matlab,
  aboveskip=3mm,
  belowskip=3mm,
  showstringspaces=false,
  columns=flexible,
  basicstyle={\small\ttfamily},
  numbers=none,
  numberstyle=\tiny\color{gray},
  keywordstyle=\color{blue},
  commentstyle=\color{dkgreen},
  stringstyle=\color{mauve},
  breaklines=true,
  breakatwhitespace=true,
  tabsize=3
}

% Macros
%%%%%%%%%%%% Text Color %%%%%%%%%%%%%%%
\definecolor{mypink1}{RGB}{219, 48, 233}
\definecolor{myred1}{RGB}{231, 76, 60}
\definecolor{myred2}{RGB}{203, 67, 53}
\definecolor{myblue1}{RGB}{52, 152, 219}
\definecolor{mygray}{gray}{0.6}

%% Table Style
\newcolumntype{C}{>{\center\arraybackslash}m{.70\columnwidth}}
\newcolumntype{Y}{>{\center\arraybackslash}m{2cm}}



%NOTE title Here
\title{Homework 6 Real Analysis}
\author{Hanyuan Zhu}



%%%%%%%%%%%%% NOTE Begin of Document %%%%%%%%%%%%%%%%%%%%%%%%%%%%%%%%%%%%%%%
%%%%%%%%%%%%%%%%%%%%%%%%%%%%%%%%%%%%%%%%%%%%%%%%%%%%%%%%%%%%%%%%%%%%%%%%%%%

\begin{document}

\maketitle

\subsection*{Problem 2.19}
\paragraph{(a)}
A and B are closed, that is the set of their limit points, $A'$ and $B'$ , are in A and B, respectively,
$ A' \subset A, B' \subset B$.

Then $ \bar{A} = A \cup A' = A $ and $ \bar{B} = B \cup B' = B$.

Since A and B are disjoint, $A \cap B = \emptyset $, then we have
\begin{equation}
  \begin{split}
    \bar{A} \cap B = A \cap B = \emptyset \\
    A \cap \bar{B} = A \cap B = \emptyset
    \end{split}
\end{equation}

Therefore A and B are separated.

\paragraph{(b)}
A and B are open and disjoint, $ A \cap B = \emptyset$.
To prove A and B are separated, that is to prove $ (\bar{A} \cap B)  \cup  (A \cap \bar{B}) = \emptyset$.

We will simplify such statement below,
\begin{equation}
  \begin{split}
    & (\bar{A} \cap B)  \cup  (A \cup \bar{B}) = \emptyset \\
    \Leftrightarrow & (\bar{A} \cap B)^c  \cap  (A \cap \bar{B})^c = \emptyset^c = X \\
    \Leftrightarrow & (\bar{A}^c \cup B^c)  \cap  (A^c \cup \bar{B}^c) = X \\
    \Leftrightarrow & (\bar{A}^c \cap A^c) \cup (\bar{A}^c \cap \bar{B}^c) \cup (B^c \cap \bar{B}^c) \cup (B^c \cap A^c) = X \\
    &\text{ because } \bar{A}^c \subset A^c \text{ and } \bar{B}^c \subset B^c \\
    \Leftrightarrow & A^c \cup (\bar{A}^c \cap \bar{B}^c) \cup B^c \cup (B^c \cap A^c) = X \\
    \Leftrightarrow & (A^c \cup B^c) \cup (\bar{A}^c \cap \bar{B}^c) \cup (B^c \cap A^c) = X \\
    &\begin{split}
      &\text{ because } \bar{A}^c \cap \bar{B}^c = (A \cup A')^c \cap (B \cup B')^c = (A^c \cap B^c) \cap (A'^c \cap B'^c)\\
      & \text{ then } \bar{A}^c \cap \bar{B}^c \subset A^c \cap B^c\\
    \end{split}\\
    \Leftrightarrow & (A^c \cup B^c) \cup (A^c \cap B^c) = X \\
    \Leftrightarrow & A^c \cup B^c = X \\
    \Leftrightarrow & A \cap B = \emptyset \\
  \end{split}
\end{equation}

According to above, if A and B are open sets  and we want to show A and B are separated, it is equivalent to show $A \cap B = \emptyset $.
Since A and B are disjoint, which is $A \cap B = \emptyset $, thus A and B are separated.

\paragraph{(c)}
For some $ p \in X$ , $\delta >0$,
$A(p) = \{ q\in X | d(p,q) < \delta \}$,
$B(p) = \{ q\in X | d(p,q) > \delta \}$.

Firstly, we show $A(p)$ and $B(p)$ are disjoint.
$$ A(p) \cap B(p) = \{ q \in X | d(p,q)< \delta \quad \& \quad d(p,q) >\delta \} = \emptyset $$
Thus  $A(p)$ and $B(p)$ are disjoint.

Secondly, we shall show $A(p)$ and $B(p)$ are open.
For every point $ q \in A(p) $,  there is $\epsilon = \frac{\delta - d(p,q)}{2}$, any $x \in N_{\epsilon}(q)$, $ x \in  A(p)$.
Therefore $A(p)$ is open.
In the same manner, let $\epsilon = \frac{d(p,q) - \delta }{2}$, we can see $B(p)$ is open.

Since $A(p)$ and $B(p)$ are disjoint and open, by part (b), we know $A(p)$ and $B(p)$ are separated.

\paragraph{(d.)}
First of all, we will show if metric space X  is connected, it is perfect.

Lets assume if X is not perfect, there is a point $x \in X$ not a limit point of X,
Therefore,  we can construct two disjoint subset A and B, such that $ A \cup B = X$, $ \bar{A} = A = \{x\}$ and $ \bar{B} = B = A^c$ .
$$ \bar{A} \cap B = \emptyset $$
$$ A \cap \bar{B} = \emptyset $$
Above means if X is nonperfect, X is not connected. It means \textbf{if metric space X  is connected, it is perfect.}

By Thm 2.43, we know a nonempty perfect set is uncountable. Therefore the connnected metric space X is uncountable.

\subsection*{Problem 2.20}
Set E is connected
\paragraph{ The closure $ \bar{E} $ is connected}
To show it , We assume there exist two disjointed subset of the closure of E, $A \cup B = \bar{E}$  , s.t. A and B are separated.

\begin{enumerate}
  \item If $A \cap E \neq \emptyset$ and $B \cap E \neq \emptyset$, Then A and B are not separated by connectedness of E.
  \item Therefore, $ A \subset E' \backslash E $ or $B \subset E'\backslash  E $. WLOG, we assume $ A \subset E' \backslash E $, Then $ \bar{B} = \bar{E}$, and $ A \cap \bar{B} = A \neq \emptyset$. So any A and B are connected
\end{enumerate}

\paragraph{ The interior $E^i$ could be separable}
Counterexample: A and B are two closed sets share some limit points in set E, and points in E are neither an interior point of A nor of B.
Such as two rectangles share a borderline.

\subsection*{Problem 2.21}

\paragraph{a.}
Firstly, we will show $ A_0 $ and $ B_0 $ are disjoint. Be definition, if $ x \in A_0 $ then $ p(x) \in A$; since A and B are seperated, $ p(x) \not\in B$ , that is $ x \not\in B_0$. And it is the same for any $x \in B_0$.

Secondly, we shall show no limit point $ x \in A'_0 $ of $ A_0$ is in $B_0$.
We assume if there exist $ x \in A'_0 $, $ x \in B_0 $. For any $\epsilon > 0$, you can find $ d(x,s) = \vert x - s \vert< \epsilon $ $ s \in A_0$.
By definition, for some $ a \in A,\text{ and }b \in B $ $ p(s) = (1-s) a + sb = a + (b-a)s \in A $ and $p(x) = (1-x)a+xb \in B$.
$d(p(x),p(s)) = \Vert p(x)-p(s) \Vert = \Vert (b-a)(x-s) \Vert = \vert x-s \vert \Vert b-a \Vert < \Vert b-a \Vert \epsilon $.
That is for any $\Vert b-a \Vert \epsilon > 0$ $ p(x)\in B $, there exist $p(s) \in A$ s.t. $ d(p(x),p(s)) < \Vert b-a \Vert \epsilon $, Therefore A has limit point in B, which contradict to that A and B are separated.
By the same means we can should no limit point of $B_0$ can be found in $A_0$.
Therefore $ A_0 $ and $ B_0 $ are separated.

\paragraph{b.} We have showed in part a. that $ A_0 $ and $ B_0 $ are separated, $ t = 0 \in A_0 $ and $ t = 1 \in B_0$, therefore  there exist $t_0 \in (0,1)$ s.t. $ t_0  \not\in A_0 \cup B_0$.
By definition, $p(t_0) \not\in A \cup B$.

\paragraph{c.} Because for any convex subset E of $R_k$, any $a \in E$ and $b \in E$, we have $p(t) \in E$, where $p(t) = (1-t)a +tb$ for all $t \in (0,1)$.
We have proved in part b. , by such condition, any nonempty subsets of E are connnected. Therefore E is connected.

\subsection*{Problem 2.24}
To show metric space X is separable, that is to show there is a countable dense subset E of X.

Before to construct such subset E, we will should X  has a finite base is bounded.
Firstly, we assume $\{ x_1, x_2 ... x_i, ... \}$ is an infinity sequence in X, and $d(x_i, x_{i+1} \leq \delta)$.
Then contradiction arises here, because such a infinity subset has no limit point, $N_{\frac{\delta}{2}} (x_i) = \{x_i\}$
Therefore the subset $\{ x_1, x_2 ... x_i, ... x_n \}$  must be finite for some n, and $ X = \cup^{n}_i N_{\delta} (x_i)$.

Thus, for some $\delta_0 > 0$, we can have a finite subset $S_{\delta_0}$, and $ |S_{\delta_0}| = N_{\delta_0} $.
By the same manner, We can construct a countable subset $ E = \cup_{n=1}^{\infty} S_{\frac{\delta_0}{n}}$, because $S_{\frac{\delta_0}{n}}$ is finite and $\{ 1, 2,3...\}$ is countable.

To show such E is dense in X. We know every subset, $S_{\frac{\delta_0}{n}}$, can make a finite cover $ \{ N_{\frac{\delta_0}{n}}(x_i) | x_i \in S_{\frac{\delta_0}{n}} \} $ of X by radius $\frac{\delta_0}{n}$.
Let $x\in X$, for arbitary $\sigma >0$ , you can have $ x \in N_{\frac{\delta_0}{n}} (x_i) $ with $ n > \frac{\delta_0}{\sigma}$, and $x_i \in  S_{\frac{\delta_0}{n}} \subset E$.
So E is dense in X.

Therefore, because there is a dense countable subset E in metric space X. X is separable.


\subsection*{Problem 2.25}
K is compact metric with a finite base. $x \in K$, $B_{\frac{1}{i}}(x)$ is the neighborhood of x with radii $\frac{1}{i}$ for some $i \in \mathbbm{N^+}$.
Thus for every  n, K has a open cover $K = \cup_{x \in  K} B_{\frac{1}{i}}(x) $, and a corresponding finite subcover is $K = \cup_{x \in E_{i}} B_{\frac{1}{i}}(x) $,
where $E_i \subset K$ is a finite subset with cardinality $|E_i| = N_i$.

Then we can construct the countable base of metric space K by $$B = \cup_{i \in \mathbbm{N^+}} \cup_{x \in E_{i}} B_{\frac{1}{i}}(x)$$
We denote the center point of the jth ball corresponding to the finite subset $E_i$ as $ x^i_j$.
The set $$ X = \{x^i_j | i \in \mathbbm{N} \text{ and } j \in {1, 2, ... N_i}\}$$ is countable, because for each i, $N_i$ is finite and $\mathbbm{N}$ is countable.

To show X is dense in K, for any $k \in K$ with arbitary $ \delta > 0$, you have $i > \frac{1}{\delta}$ s.t. $ k \in B_{\frac{1}{i}}(x)^i_j$ with some $j < N_i$.

Therefore X is a countable dense subset of K, which means K is separable.

\subsection*{Problem 2.29}
First, we want to prove for every open set $E \subset R^1$, there exists a union of disjoint segments s.t. $ \cup_i S_i = E$.

For any $ x \in E $, we denote $ \{S^{(x)}_{\alpha}\} $ as a collection of intervals , $S^{(x)}_{\alpha} \subset E$, that contains $x$, and $S^{(x)} = \cup_{\alpha} S^{(x)}_{\alpha}$.
Such $S^{x}$ is the maximum segments contains $x$, which means if $ y \in S^{x}$, $S^{x} = S^{y}$ and if $ y \not\in S^{x}$, $S^{x} \cap S^{x} = \emptyset$ that is disjointed. For all $x \in E$, we will have a collection of distinct $T = \{S^{(x)}\} $.
$ E = \cup_{T} S^{(x)}$

Secondly, we will show such T is a countable collection. Since rational number is dense in $R^1$,  any $ S^{(x)} \subset E$ contains rational numbers. Since every disjoint(or distinct) $S^{(x)}$ contains distinct rational numbers.
If T is a uncountable collection, $\{S^{(x)}_{\alpha}\} $, it means there are uncountably many distinct rational numbers in E, which is a contradiction to the countability of rational number.
Therefore open set E is a union of at most countable collection of disjoint segments.

\subsection*{Problem 3.1}
\paragraph{a.}
Since $\{ s_n \}$ converges, it implies, for every $ \epsilon >0$, there exists N s.t. $|s_n - s_m| < \epsilon$ for any $m>N$ and $n >N$.
$ ||s_n|- |s_m|| < |s_n - s_m| < \epsilon $ for any $m>N$ and $n >N$. Thus $ \{ |s_n| \}$ converges.

\paragraph{b.} When $s_n = (-1)^n$ ,  $ \{ |s_n| \}$ converge 1, but $\{ s_n \}$  doesn't converge.

\subsection*{Problem 3.2}

\begin{equation}
  \begin{split}
    \lim_{n \rightarrow \infty}  \sqrt{n^2 + n} - n & = \lim_{n \rightarrow \infty} \frac{(\sqrt{n^2 + n} - n)(\sqrt{n^2 + n} + n)}{\sqrt{n^2 + n} + n} \\
    & = \lim_{n \rightarrow \infty} \frac{n^2 + n - n^2}{\sqrt{n^2 + n} + n} \\
    & = \lim_{n \rightarrow \infty} \frac{ n }{\sqrt{n^2 + n} + n} \\
    & = \lim_{n \rightarrow \infty} \frac{ 1 }{\sqrt{1 + \frac{1}{n}} + 1} \\
    & = \frac{1}{2}
  \end{split}
\end{equation}

\subsection*{Problem 3.3}
\paragraph{ Firstly, we show $ \{s_n\}$ is bounded. }

Since $0 < s_1 = \sqrt{2} < 2$, we assume $ 0< s_n < 2$. $$ 0 < s_{n+1} = \sqrt{2 + \sqrt{s_n}} < \sqrt{2 + \sqrt{2}} < \sqrt{2 + 2} = 2 $$
Therefore $ \{s_n\}$ is bounded by 0 and 2.

\paragraph{ Secondly, we show $ \{s_n\}$ is monotonic.}

\begin{equation}
  s_2 =  \sqrt{2 + \sqrt{2}} > \sqrt{2} = s_1
\end{equation}

Assmue $s_n > s_{n-1}$.

\begin{equation}
  s_n+1 =  \sqrt{2 + \sqrt{s_n}} > \sqrt{2 + \sqrt{s_{n-1}}} = s_n
\end{equation}

Therefore $ \{s_n\}$ is monotonically increase.

By theorem 3.14, $ \{s_n\}$ converges.

\end{document}

\documentclass{article}

\usepackage{listings}
\usepackage{color}

\definecolor{dkgreen}{rgb}{0,0.6,0}
\definecolor{gray}{rgb}{0.5,0.5,0.5}
\definecolor{mauve}{rgb}{0.58,0,0.82}

\lstset{frame=tb,
  language=Python,
  aboveskip=3mm,
  belowskip=3mm,
  showstringspaces=false,
  columns=flexible,
  basicstyle={\small\ttfamily},
  numbers=none,
  numberstyle=\tiny\color{gray},
  keywordstyle=\color{blue},
  commentstyle=\color{dkgreen},
  stringstyle=\color{mauve},
  breaklines=true,
  breakatwhitespace=true,
  tabsize=3
}



\begin{document}

\section*{Question 5}
\begin{enumerate}
  \item The Principle Components are $T = XU$ where $X$ are the data set and
  \begin{equation}
  U = \bigg[ \begin{tabular}{c c}
  -0.7930 &   0.6093 \\
  0.6093  &  0.7930 \\
\end{tabular}
 \bigg]
\end{equation}
\item There is no dominant linear direction in the data, since the ratio of two eigenvalues of covariance matrix is about 1.
  \end{enumerate}

  \section*{Question 6}
  \begin{enumerate}
    \item The Principle Components are $T = XU$ where $X$ are the data set and
    \begin{equation}
    U = \bigg[ \begin{tabular}{c c}
    -1.0000  &  0.0070 \\
 0.0070  &  1.0000 \\
  \end{tabular}
   \bigg]
  \end{equation}
  You can see clearly the data are stretched most along the first primary component direction.
  \item The dominant direction is along $[0.0070,1.0000]$ with the primary eigenvalue of $0.9672$, which is much larger than the second eigenvalue $0.0630$.
    \end{enumerate}

\section*{Code for Questions}
\begin{tabular}{c|c}

\end{tabular}
\begin{lstlisting}
%% Sample ~1000 points uniformly from a ball

X_Circle=2*rand(floor(1000*(4/pi)),2)-[1,1]; % Generate data uniformly at random
X_Circle=X_Circle(sum(X_Circle.^2,2)<1,:); %Use rejection sampling to get points to live in ball

%%First Principle component
Cov_Circle = cov(X_Circle);
[V_C,D_C] = eig(Cov_Circle);
%%PCA
PC_1 = X_Circle * V_C(:,1);
PC_2 = X_Circle * V_C(:,2);
length = zeros(size(PC_1,1),1);
close all; % Display data
%scatter(X_Circle(:,1),X_Circle(:,2));
scatter(length,PC_1);
axis equal
title('Data Generated Uniformly From a Ball of Radius 1','Interpreter','Latex');

%% Sample ~1000 points uniformly from an ellipse with major axis lengths 2,.5.


X_Ellipse=4*rand(floor(1000*(16/pi)),2)-[2,2];
X_Ellipse=X_Ellipse(sum([4,.25].*X_Ellipse.^2,2)<1,:);
%%First Principle component
Cov_Ellipse = cov(X_Ellipse);
[V_E,D_E] = eig(Cov_Ellipse);
%%Principle Components
PC_E1 = X_Ellipse * V_E(:,1);
PC_E2 = X_Ellipse * V_E(:,2);
length_E = zeros(size(PC_E1,1),1);

close all;
scatter(X_Ellipse(:,1),X_Ellipse(:,2));


axis equal
title('Data Generated Uniformly From Interior of the Ellipse $4x^{2}+\frac{y^{2}}{4}=1$','Interpreter','latex');
         \end{lstlisting}



\end{document}

\documentclass{article}

%change the margin of the paper
%\usepackage[legalpaper, margin=0.1in]{geometry}
%using the \substack
\usepackage{amsmath}
% \mathbbm{}
\usepackage{bbm}

% NOTE \pdv  partial differential equation
\usepackage{physics}

% hyperlink Here
\usepackage{hyperref}
\hypersetup{
    colorlinks=true,
    linkcolor=blue,
    filecolor=magenta,
    urlcolor=cyan,
}

% \cancelto{0}{A} to cross the A to 0
\usepackage{cancel}


% this is the package for block comment \begin{comment} and \end{comment}
\usepackage{verbatim}
\usepackage{imakeidx}
% For multiple rows in tabular environment.
\usepackage{multirow}
% use this package to strikeout the word /st{}
%the color package is for the \textcolor{red} to highlight the text
\usepackage{color,soul}
% to define newcolumntype and \arraybackslash
\usepackage{array}
% hyperref is to call /url. hyphen packege to avoid that the url is too long
\PassOptionsToPackage{hyphens}{url}\usepackage{hyperref}
%Todo list, \newlist \setlist...
\usepackage{enumitem,amssymb}
\newlist{todolist}{itemize}{2}
\setlist[todolist]{label=$\square$}

% for image inserting
\usepackage{graphicx}
\graphicspath{./Desktop/Homework/123/7/}
\usepackage{subfig}

% \iff \leqlsant
\usepackage{amssymb}



% Code block begin(lstlisting) and end(lstlisting)
\usepackage{listings}
\usepackage{color}

\definecolor{dkgreen}{rgb}{0,0.6,0}
\definecolor{gray}{rgb}{0.5,0.5,0.5}
\definecolor{mauve}{rgb}{0.58,0,0.82}


%NOTE: Change the "language" parameter here
\lstset{frame=tb,
  language=Matlab,
  aboveskip=3mm,
  belowskip=3mm,
  showstringspaces=false,
  columns=flexible,
  basicstyle={\small\ttfamily},
  numbers=none,
  numberstyle=\tiny\color{gray},
  keywordstyle=\color{blue},
  commentstyle=\color{dkgreen},
  stringstyle=\color{mauve},
  breaklines=true,
  breakatwhitespace=true,
  tabsize=3
}

% Macros
%%%%%%%%%%%% Text Color %%%%%%%%%%%%%%%
\definecolor{mypink1}{RGB}{219, 48, 233}
\definecolor{myred1}{RGB}{231, 76, 60}
\definecolor{myred2}{RGB}{203, 67, 53}
\definecolor{myblue1}{RGB}{52, 152, 219}
\definecolor{mygray}{gray}{0.6}

%% Table Style
\newcolumntype{C}{>{\center\arraybackslash}m{.70\columnwidth}}
\newcolumntype{Y}{>{\center\arraybackslash}m{2cm}}



%NOTE title Here
\title{Homework 8 Numerical Analysis}
\author{Hanyuan Zhu}
\date{}



%%%%%%%%%%%%% NOTE Begin of Document %%%%%%%%%%%%%%%%%%%%%%%%%%%%%%%%%%%%%%%
%%%%%%%%%%%%%%%%%%%%%%%%%%%%%%%%%%%%%%%%%%%%%%%%%%%%%%%%%%%%%%%%%%%%%%%%%%%

\begin{document}

\section*{Question2}


\subsection*{2a. }

\begin{equation}
  \begin{split}
    y_i (w^T x_i + b) \geq 1 - \xi_i\\
    \xi_i \geq 1 - y_i (w^T x_i + b)\\
  \end{split}
\end{equation}

If $1 - y_i (w^T x_i + b) > 0 $, $\xi_i^2 \geq (1 - y_i (w^T x_i + b))^2 $ ;\\
if $1 - y_i (w^T x_i + b) < 0 $, $ |\xi_i| > 0 >  1 - y_i (w^T x_i + b) $.

\begin{equation}
  |\xi_i|^2 \geq
  \begin{cases}
    (1 - y_i (w^T x_i + b))^2, &  y_i (w^T x_i + b) < 1 \\
    0 , & y_i (w^T x_i + b)>1
  \end{cases}
\end{equation}

$$ \Vert \xi \Vert^2 = \sum_i|\xi_i|^2  $$

Substitute above into $G(w, b, \xi)$, we have

\begin{equation}
  \begin{split}
    G(w, b, \xi) &= \Vert w \Vert^2 + \alpha  \Vert \xi \Vert^2\\
    &= \Vert w \Vert^2 + \alpha \sum_i|\xi_i|^2\\
    &= \Vert w \Vert^2 + \alpha \sum_i \max(0,(1 - y_i (w^T x_i + b)))^2\\
    &= F(w, b)
  \end{split}
\end{equation}

\subsection*{2b. }

We construct the Lagrangian for our problem
\begin{equation}
  \begin{split}
    L(w, b, \alpha) &= \Vert w \Vert^2 + \alpha \Vert \xi \Vert^2 + \sum_i \lumbda_i (1 - y_i (w^T x_i + b))^2\\
    &= \frac{1}{2}\Vert w \Vert^2 + \sum_i \alpha_i (1 + y_i^2 (w^T x_i + b)^2 - 2 y_i (w^T x_i + b))\\
    &= \frac{1}{2}\Vert w \Vert^2 + \sum_i \alpha_i (1 + y_i^2 (\Vert w \Vert^2 \Vert x_i \Vert^2 + b^2 + 2 b x_i^T w  ) - 2 y_i ( x_i^T w + b))\\
    \pdv{L}{w} &=  w +  \sum_i \alpha_i ( y_i^2 (2\Vert x_i \Vert^2 w + 2 b x_i^T )- 2 y_i x_i ^ T ) = 0 \\
    \sum_i  w ( \frac{1}{n} + 2 \alpha_i &( y_i^2 \Vert x_i \Vert^2 ) = \sum_i 2 \alpha_i y_i x_i ^ T ( 1 - y_i b  )\\
    w ( \frac{1}{n} + 2 \alpha_i &( y_i^2 \Vert x_i \Vert^2 ) = 2 \alpha_i y_i x_i ^ T ( 1 - y_i b  )\\
  \end{split}
\end{equation}



\section*{Question3}
\subsection*{3a. }
Suppose if there is only one point as a support vector, that means all geometric margin are larger then the minimum geometric margin $ \gamma_0 = \min \gamma_i = \min y_i (w^T x_i + b) $, except one point $(y_0,x_0)$.
Since the hyperplane bisect the data, there is a point on different side such that, for $y_i \neq y_0 $, $ \gamma_1 = \min \gamma_i = \min y_i (w^T x_i + b) $. Then we have $ \gamma_1 +  \gamma_0 > 2  \gamma_0$ , which means the hyperplane is not margin-maximal.

\subsection*{3b. }
Yes, there could be more than 2 support vectors. In a 2-D data space, 3 data points which can form a isosceles triangle, then all three points will be the support vectors.


\end{document}
